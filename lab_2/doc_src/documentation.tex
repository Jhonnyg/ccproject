%
%  documentation
%
%  Created by work on 2010-04-16.
%  Copyright (c) 2010 . All rights reserved.
%
\documentclass[]{article}

% Use utf-8 encoding for foreign characters
\usepackage[utf8]{inputenc}

% Setup for fullpage use
\usepackage{fullpage}

% Uncomment some of the following if you use the features
%
% Running Headers and footers
%\usepackage{fancyhdr}

% Multipart figures
%\usepackage{subfigure}

% More symbols
%\usepackage{amsmath}
%\usepackage{amssymb}
%\usepackage{latexsym}

% Surround parts of graphics with box
\usepackage{boxedminipage}

% Package for including code in the document
\usepackage{listings}

% If you want to generate a toc for each chapter (use with book)
\usepackage{minitoc}

% This is now the recommended way for checking for PDFLaTeX:
\usepackage{ifpdf}

%\newif\ifpdf
%\ifx\pdfoutput\undefined
%\pdffalse % we are not running PDFLaTeX
%\else
%\pdfoutput=1 % we are running PDFLaTeX
%\pdftrue
%\fi

\ifpdf
\usepackage[pdftex]{graphicx}
\else
\usepackage{graphicx}
\fi
\title{JLC - A Javalette Compiler}
\author{Sven Andersson\\19860708-4632 -- andsve@student.chalmers.se
        \and Jhonny Göransson\\19840611-8235 -- jhonny@student.chalmers.se}

\date{2010-04-16}

\begin{document}

\ifpdf
\DeclareGraphicsExtensions{.pdf, .jpg, .tif}
\else
\DeclareGraphicsExtensions{.eps, .jpg}
\fi

\maketitle
\begin{center}
  Revision 1
\end{center}


%\begin{abstract}
%\end{abstract}

\section{The JLC Compiler}
  \subsection*{Building}
    To build the JLC binary execute;\\
    \texttt{\$ cd src/}\\
    \texttt{\$ make}\\
    It will produce a \texttt{jlc} binary in the project root directory.\\
    \\
    The following command will remove all files produced during building;\\
    \texttt{\$ make clean}

  \subsection*{Usage}
    \texttt{\$ ./jlc \textit{filename.jl}}\\
    Will generate \texttt{filename.j} (Jasmin file) and \texttt{filename.class} (Java Class file).
  
  \subsection*{How it works}
    The compiler executes in the following 4 steps:
    \begin{enumerate}
      \item \textbf{Lexer} -- Generates a code tree from the source file.
      \item \textbf{Typechecker} -- Checks the code tree for type errors.
      \item \textbf{(Jasmin) Compiler} -- Generates a \texttt{.j} file with Jasmin\footnote{http://jasmin.sourceforge.net/} instructions
      \item \textbf{Java Class generation} -- The compiler automatically runs the Jasmin binary (\texttt{jasmin.jar}) located in the subdirectory \texttt{lib}, which generates a Java Class file from the \texttt{.j} file.
    \end{enumerate}

\section{BNFC Results}
  We get one shift/reduce conflict from BNFC, which comes from the "dangling" if/if-else rule.

\section{The Javalette Language}
\batchmode
%This Latex file is machine-generated by the BNF-converter

\documentclass[a4paper,11pt]{article}
\author{BNF-converter}
\title{The Language javalette}
\setlength{\parindent}{0mm}
\setlength{\parskip}{1mm}
\begin{document}

\maketitle

\newcommand{\emptyP}{\mbox{$\epsilon$}}
\newcommand{\terminal}[1]{\mbox{{\texttt {#1}}}}
\newcommand{\nonterminal}[1]{\mbox{$\langle \mbox{{\sl #1 }} \! \rangle$}}
\newcommand{\arrow}{\mbox{::=}}
\newcommand{\delimit}{\mbox{$|$}}
\newcommand{\reserved}[1]{\mbox{{\texttt {#1}}}}
\newcommand{\literal}[1]{\mbox{{\texttt {#1}}}}
\newcommand{\symb}[1]{\mbox{{\texttt {#1}}}}

This document was automatically generated by the {\em BNF-Converter}. It was generated together with the lexer, the parser, and the abstract syntax module, which guarantees that the document matches with the implementation of the language (provided no hand-hacking has taken place).

\section*{The lexical structure of javalette}
\subsection*{Identifiers}
Identifiers \nonterminal{Ident} are unquoted strings beginning with a letter,
followed by any combination of letters, digits, and the characters {\tt \_ '},
reserved words excluded.


\subsection*{Literals}
Integer literals \nonterminal{Int}\ are nonempty sequences of digits.


Double-precision float literals \nonterminal{Double}\ have the structure
indicated by the regular expression $\nonterminal{digit}+ \mbox{{\it `.'}} \nonterminal{digit}+ (\mbox{{\it `e'}} \mbox{{\it `-'}}? \nonterminal{digit}+)?$ i.e.\
two sequences of digits separated by a decimal point, optionally
followed by an unsigned or negative exponent.


String literals \nonterminal{String}\ have the form
\terminal{"}$x$\terminal{"}, where $x$ is any sequence of any characters
except \terminal{"}\ unless preceded by \verb6\6.




\subsection*{Reserved words and symbols}
The set of reserved words is the set of terminals appearing in the grammar. Those reserved words that consist of non-letter characters are called symbols, and they are treated in a different way from those that are similar to identifiers. The lexer follows rules familiar from languages like Haskell, C, and Java, including longest match and spacing conventions.

The reserved words used in javalette are the following: \\

\begin{tabular}{lll}
{\reserved{boolean}} &{\reserved{double}} &{\reserved{else}} \\
{\reserved{false}} &{\reserved{if}} &{\reserved{int}} \\
{\reserved{return}} &{\reserved{true}} &{\reserved{void}} \\
{\reserved{while}} & & \\
\end{tabular}\\

The symbols used in javalette are the following: \\

\begin{tabular}{lll}
{\symb{(}} &{\symb{)}} &{\symb{,}} \\
{\symb{\{}} &{\symb{\}}} &{\symb{;}} \\
{\symb{{$=$}}} &{\symb{{$+$}{$+$}}} &{\symb{{$-$}{$-$}}} \\
{\symb{{$-$}}} &{\symb{!}} &{\symb{\&\&}} \\
{\symb{{$|$}{$|$}}} &{\symb{{$+$}}} &{\symb{*}} \\
{\symb{/}} &{\symb{\%}} &{\symb{{$<$}}} \\
{\symb{{$<$}{$=$}}} &{\symb{{$>$}}} &{\symb{{$>$}{$=$}}} \\
{\symb{{$=$}{$=$}}} &{\symb{!{$=$}}} & \\
\end{tabular}\\

\subsection*{Comments}
Single-line comments begin with {\symb{\#}}, {\symb{//}}. \\Multiple-line comments are  enclosed with {\symb{/*}} and {\symb{*/}}.

\section*{The syntactic structure of javalette}
Non-terminals are enclosed between $\langle$ and $\rangle$. 
The symbols  {\arrow}  (production),  {\delimit}  (union) 
and {\emptyP} (empty rule) belong to the BNF notation. 
All other symbols are terminals.\\

\begin{tabular}{lll}
{\nonterminal{Program}} & {\arrow}  &{\nonterminal{ListTopDef}}  \\
\end{tabular}\\

\begin{tabular}{lll}
{\nonterminal{TopDef}} & {\arrow}  &{\nonterminal{Type}} {\nonterminal{Ident}} {\terminal{(}} {\nonterminal{ListArg}} {\terminal{)}} {\nonterminal{Block}}  \\
\end{tabular}\\

\begin{tabular}{lll}
{\nonterminal{ListTopDef}} & {\arrow}  &{\nonterminal{TopDef}}  \\
 & {\delimit}  &{\nonterminal{TopDef}} {\nonterminal{ListTopDef}}  \\
\end{tabular}\\

\begin{tabular}{lll}
{\nonterminal{Arg}} & {\arrow}  &{\nonterminal{Type}} {\nonterminal{Ident}}  \\
\end{tabular}\\

\begin{tabular}{lll}
{\nonterminal{ListArg}} & {\arrow}  &{\emptyP} \\
 & {\delimit}  &{\nonterminal{Arg}}  \\
 & {\delimit}  &{\nonterminal{Arg}} {\terminal{,}} {\nonterminal{ListArg}}  \\
\end{tabular}\\

\begin{tabular}{lll}
{\nonterminal{Stmt}} & {\arrow}  &{\terminal{;}}  \\
 & {\delimit}  &{\nonterminal{Block}}  \\
 & {\delimit}  &{\nonterminal{Type}} {\nonterminal{ListItem}} {\terminal{;}}  \\
 & {\delimit}  &{\nonterminal{Ident}} {\terminal{{$=$}}} {\nonterminal{Expr}} {\terminal{;}}  \\
 & {\delimit}  &{\nonterminal{Ident}} {\terminal{{$+$}{$+$}}} {\terminal{;}}  \\
 & {\delimit}  &{\nonterminal{Ident}} {\terminal{{$-$}{$-$}}} {\terminal{;}}  \\
 & {\delimit}  &{\terminal{return}} {\nonterminal{Expr}} {\terminal{;}}  \\
 & {\delimit}  &{\terminal{return}} {\terminal{;}}  \\
 & {\delimit}  &{\terminal{if}} {\terminal{(}} {\nonterminal{Expr}} {\terminal{)}} {\nonterminal{Stmt}}  \\
 & {\delimit}  &{\terminal{if}} {\terminal{(}} {\nonterminal{Expr}} {\terminal{)}} {\nonterminal{Stmt}} {\terminal{else}} {\nonterminal{Stmt}}  \\
 & {\delimit}  &{\terminal{while}} {\terminal{(}} {\nonterminal{Expr}} {\terminal{)}} {\nonterminal{Stmt}}  \\
 & {\delimit}  &{\nonterminal{Expr}} {\terminal{;}}  \\
\end{tabular}\\

\begin{tabular}{lll}
{\nonterminal{Block}} & {\arrow}  &{\terminal{\{}} {\nonterminal{ListStmt}} {\terminal{\}}}  \\
\end{tabular}\\

\begin{tabular}{lll}
{\nonterminal{ListStmt}} & {\arrow}  &{\emptyP} \\
 & {\delimit}  &{\nonterminal{Stmt}} {\nonterminal{ListStmt}}  \\
\end{tabular}\\

\begin{tabular}{lll}
{\nonterminal{Item}} & {\arrow}  &{\nonterminal{Ident}}  \\
 & {\delimit}  &{\nonterminal{Ident}} {\terminal{{$=$}}} {\nonterminal{Expr}}  \\
\end{tabular}\\

\begin{tabular}{lll}
{\nonterminal{ListItem}} & {\arrow}  &{\nonterminal{Item}}  \\
 & {\delimit}  &{\nonterminal{Item}} {\terminal{,}} {\nonterminal{ListItem}}  \\
\end{tabular}\\

\begin{tabular}{lll}
{\nonterminal{Type}} & {\arrow}  &{\terminal{int}}  \\
 & {\delimit}  &{\terminal{double}}  \\
 & {\delimit}  &{\terminal{boolean}}  \\
 & {\delimit}  &{\terminal{void}}  \\
\end{tabular}\\

\begin{tabular}{lll}
{\nonterminal{ListType}} & {\arrow}  &{\emptyP} \\
 & {\delimit}  &{\nonterminal{Type}}  \\
 & {\delimit}  &{\nonterminal{Type}} {\terminal{,}} {\nonterminal{ListType}}  \\
\end{tabular}\\

\begin{tabular}{lll}
{\nonterminal{Expr6}} & {\arrow}  &{\nonterminal{Ident}}  \\
 & {\delimit}  &{\nonterminal{Integer}}  \\
 & {\delimit}  &{\nonterminal{Double}}  \\
 & {\delimit}  &{\terminal{true}}  \\
 & {\delimit}  &{\terminal{false}}  \\
 & {\delimit}  &{\nonterminal{Ident}} {\terminal{(}} {\nonterminal{ListExpr}} {\terminal{)}}  \\
 & {\delimit}  &{\nonterminal{Ident}} {\terminal{(}} {\nonterminal{String}} {\terminal{)}}  \\
 & {\delimit}  &{\terminal{(}} {\nonterminal{Expr}} {\terminal{)}}  \\
\end{tabular}\\

\begin{tabular}{lll}
{\nonterminal{Expr5}} & {\arrow}  &{\terminal{{$-$}}} {\nonterminal{Expr6}}  \\
 & {\delimit}  &{\terminal{!}} {\nonterminal{Expr6}}  \\
 & {\delimit}  &{\nonterminal{Expr6}}  \\
\end{tabular}\\

\begin{tabular}{lll}
{\nonterminal{Expr4}} & {\arrow}  &{\nonterminal{Expr4}} {\nonterminal{MulOp}} {\nonterminal{Expr5}}  \\
 & {\delimit}  &{\nonterminal{Expr5}}  \\
\end{tabular}\\

\begin{tabular}{lll}
{\nonterminal{Expr3}} & {\arrow}  &{\nonterminal{Expr3}} {\nonterminal{AddOp}} {\nonterminal{Expr4}}  \\
 & {\delimit}  &{\nonterminal{Expr4}}  \\
\end{tabular}\\

\begin{tabular}{lll}
{\nonterminal{Expr2}} & {\arrow}  &{\nonterminal{Expr2}} {\nonterminal{RelOp}} {\nonterminal{Expr3}}  \\
 & {\delimit}  &{\nonterminal{Expr3}}  \\
\end{tabular}\\

\begin{tabular}{lll}
{\nonterminal{Expr1}} & {\arrow}  &{\nonterminal{Expr2}} {\terminal{\&\&}} {\nonterminal{Expr1}}  \\
 & {\delimit}  &{\nonterminal{Expr2}}  \\
\end{tabular}\\

\begin{tabular}{lll}
{\nonterminal{Expr}} & {\arrow}  &{\nonterminal{Expr1}} {\terminal{{$|$}{$|$}}} {\nonterminal{Expr}}  \\
 & {\delimit}  &{\nonterminal{Expr1}}  \\
\end{tabular}\\

\begin{tabular}{lll}
{\nonterminal{ListExpr}} & {\arrow}  &{\emptyP} \\
 & {\delimit}  &{\nonterminal{Expr}}  \\
 & {\delimit}  &{\nonterminal{Expr}} {\terminal{,}} {\nonterminal{ListExpr}}  \\
\end{tabular}\\

\begin{tabular}{lll}
{\nonterminal{AddOp}} & {\arrow}  &{\terminal{{$+$}}}  \\
 & {\delimit}  &{\terminal{{$-$}}}  \\
\end{tabular}\\

\begin{tabular}{lll}
{\nonterminal{MulOp}} & {\arrow}  &{\terminal{*}}  \\
 & {\delimit}  &{\terminal{/}}  \\
 & {\delimit}  &{\terminal{\%}}  \\
\end{tabular}\\

\begin{tabular}{lll}
{\nonterminal{RelOp}} & {\arrow}  &{\terminal{{$<$}}}  \\
 & {\delimit}  &{\terminal{{$<$}{$=$}}}  \\
 & {\delimit}  &{\terminal{{$>$}}}  \\
 & {\delimit}  &{\terminal{{$>$}{$=$}}}  \\
 & {\delimit}  &{\terminal{{$=$}{$=$}}}  \\
 & {\delimit}  &{\terminal{!{$=$}}}  \\
\end{tabular}\\



\end{document}



\bibliographystyle{plain}
\bibliography{}
\end{document}
